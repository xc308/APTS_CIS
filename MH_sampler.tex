% Options for packages loaded elsewhere
\PassOptionsToPackage{unicode}{hyperref}
\PassOptionsToPackage{hyphens}{url}
%
\documentclass[
]{article}
\usepackage{amsmath,amssymb}
\usepackage{lmodern}
\usepackage{ifxetex,ifluatex}
\ifnum 0\ifxetex 1\fi\ifluatex 1\fi=0 % if pdftex
  \usepackage[T1]{fontenc}
  \usepackage[utf8]{inputenc}
  \usepackage{textcomp} % provide euro and other symbols
\else % if luatex or xetex
  \usepackage{unicode-math}
  \defaultfontfeatures{Scale=MatchLowercase}
  \defaultfontfeatures[\rmfamily]{Ligatures=TeX,Scale=1}
\fi
% Use upquote if available, for straight quotes in verbatim environments
\IfFileExists{upquote.sty}{\usepackage{upquote}}{}
\IfFileExists{microtype.sty}{% use microtype if available
  \usepackage[]{microtype}
  \UseMicrotypeSet[protrusion]{basicmath} % disable protrusion for tt fonts
}{}
\makeatletter
\@ifundefined{KOMAClassName}{% if non-KOMA class
  \IfFileExists{parskip.sty}{%
    \usepackage{parskip}
  }{% else
    \setlength{\parindent}{0pt}
    \setlength{\parskip}{6pt plus 2pt minus 1pt}}
}{% if KOMA class
  \KOMAoptions{parskip=half}}
\makeatother
\usepackage{xcolor}
\IfFileExists{xurl.sty}{\usepackage{xurl}}{} % add URL line breaks if available
\IfFileExists{bookmark.sty}{\usepackage{bookmark}}{\usepackage{hyperref}}
\hypersetup{
  pdftitle={MHsampler},
  pdfauthor={XC},
  hidelinks,
  pdfcreator={LaTeX via pandoc}}
\urlstyle{same} % disable monospaced font for URLs
\usepackage[margin=1in]{geometry}
\usepackage{color}
\usepackage{fancyvrb}
\newcommand{\VerbBar}{|}
\newcommand{\VERB}{\Verb[commandchars=\\\{\}]}
\DefineVerbatimEnvironment{Highlighting}{Verbatim}{commandchars=\\\{\}}
% Add ',fontsize=\small' for more characters per line
\usepackage{framed}
\definecolor{shadecolor}{RGB}{248,248,248}
\newenvironment{Shaded}{\begin{snugshade}}{\end{snugshade}}
\newcommand{\AlertTok}[1]{\textcolor[rgb]{0.94,0.16,0.16}{#1}}
\newcommand{\AnnotationTok}[1]{\textcolor[rgb]{0.56,0.35,0.01}{\textbf{\textit{#1}}}}
\newcommand{\AttributeTok}[1]{\textcolor[rgb]{0.77,0.63,0.00}{#1}}
\newcommand{\BaseNTok}[1]{\textcolor[rgb]{0.00,0.00,0.81}{#1}}
\newcommand{\BuiltInTok}[1]{#1}
\newcommand{\CharTok}[1]{\textcolor[rgb]{0.31,0.60,0.02}{#1}}
\newcommand{\CommentTok}[1]{\textcolor[rgb]{0.56,0.35,0.01}{\textit{#1}}}
\newcommand{\CommentVarTok}[1]{\textcolor[rgb]{0.56,0.35,0.01}{\textbf{\textit{#1}}}}
\newcommand{\ConstantTok}[1]{\textcolor[rgb]{0.00,0.00,0.00}{#1}}
\newcommand{\ControlFlowTok}[1]{\textcolor[rgb]{0.13,0.29,0.53}{\textbf{#1}}}
\newcommand{\DataTypeTok}[1]{\textcolor[rgb]{0.13,0.29,0.53}{#1}}
\newcommand{\DecValTok}[1]{\textcolor[rgb]{0.00,0.00,0.81}{#1}}
\newcommand{\DocumentationTok}[1]{\textcolor[rgb]{0.56,0.35,0.01}{\textbf{\textit{#1}}}}
\newcommand{\ErrorTok}[1]{\textcolor[rgb]{0.64,0.00,0.00}{\textbf{#1}}}
\newcommand{\ExtensionTok}[1]{#1}
\newcommand{\FloatTok}[1]{\textcolor[rgb]{0.00,0.00,0.81}{#1}}
\newcommand{\FunctionTok}[1]{\textcolor[rgb]{0.00,0.00,0.00}{#1}}
\newcommand{\ImportTok}[1]{#1}
\newcommand{\InformationTok}[1]{\textcolor[rgb]{0.56,0.35,0.01}{\textbf{\textit{#1}}}}
\newcommand{\KeywordTok}[1]{\textcolor[rgb]{0.13,0.29,0.53}{\textbf{#1}}}
\newcommand{\NormalTok}[1]{#1}
\newcommand{\OperatorTok}[1]{\textcolor[rgb]{0.81,0.36,0.00}{\textbf{#1}}}
\newcommand{\OtherTok}[1]{\textcolor[rgb]{0.56,0.35,0.01}{#1}}
\newcommand{\PreprocessorTok}[1]{\textcolor[rgb]{0.56,0.35,0.01}{\textit{#1}}}
\newcommand{\RegionMarkerTok}[1]{#1}
\newcommand{\SpecialCharTok}[1]{\textcolor[rgb]{0.00,0.00,0.00}{#1}}
\newcommand{\SpecialStringTok}[1]{\textcolor[rgb]{0.31,0.60,0.02}{#1}}
\newcommand{\StringTok}[1]{\textcolor[rgb]{0.31,0.60,0.02}{#1}}
\newcommand{\VariableTok}[1]{\textcolor[rgb]{0.00,0.00,0.00}{#1}}
\newcommand{\VerbatimStringTok}[1]{\textcolor[rgb]{0.31,0.60,0.02}{#1}}
\newcommand{\WarningTok}[1]{\textcolor[rgb]{0.56,0.35,0.01}{\textbf{\textit{#1}}}}
\usepackage{graphicx}
\makeatletter
\def\maxwidth{\ifdim\Gin@nat@width>\linewidth\linewidth\else\Gin@nat@width\fi}
\def\maxheight{\ifdim\Gin@nat@height>\textheight\textheight\else\Gin@nat@height\fi}
\makeatother
% Scale images if necessary, so that they will not overflow the page
% margins by default, and it is still possible to overwrite the defaults
% using explicit options in \includegraphics[width, height, ...]{}
\setkeys{Gin}{width=\maxwidth,height=\maxheight,keepaspectratio}
% Set default figure placement to htbp
\makeatletter
\def\fps@figure{htbp}
\makeatother
\setlength{\emergencystretch}{3em} % prevent overfull lines
\providecommand{\tightlist}{%
  \setlength{\itemsep}{0pt}\setlength{\parskip}{0pt}}
\setcounter{secnumdepth}{-\maxdimen} % remove section numbering
\ifluatex
  \usepackage{selnolig}  % disable illegal ligatures
\fi

\title{MHsampler}
\author{XC}
\date{19/07/2021}

\begin{document}
\maketitle

\hypertarget{metropolishastings-algorithms}{%
\subsection{Metropolis--Hastings
Algorithms}\label{metropolishastings-algorithms}}

Unlike IS sampling generating iid samples, MH generates correlated
variables from MC, but it provide easier proposal when IS doesn't apply.

\begin{itemize}
\item
  only little need to be known about the target f
\item
  Markovian property leads to efficient decompositions of
  high-dimensional problems in a sequence of smaller problems that are
  much easier to solve
\item
  Most of your time and energy will be spent in designing and assessing
  your MCMC algorithms
\item
  Incredible feature of MH:

  \begin{itemize}
  \tightlist
  \item
    for every given q(.), we can construct a Metropolis-Hasting kernel
    whose stationary is the target f
  \item
  \end{itemize}
\end{itemize}

\hypertarget{example-6.1}{%
\subsection{Example 6.1}\label{example-6.1}}

\begin{itemize}
\tightlist
\item
  Target distriubtion: Beta(2.7, 6.3)
\item
  proposal dist q: unif{[}0,1{]}, means does not depend on previous step
  value of the chain
\item
  MH algo
\end{itemize}

\begin{Shaded}
\begin{Highlighting}[]
\NormalTok{a }\OtherTok{\textless{}{-}} \FloatTok{2.7}\NormalTok{; b }\OtherTok{=} \FloatTok{6.3}

\NormalTok{Nsim }\OtherTok{\textless{}{-}} \DecValTok{5000}
\NormalTok{X }\OtherTok{\textless{}{-}} \FunctionTok{rep}\NormalTok{(}\FunctionTok{runif}\NormalTok{(}\DecValTok{1}\NormalTok{), Nsim)   }\CommentTok{\# initialize the chain}
\ControlFlowTok{for}\NormalTok{(i }\ControlFlowTok{in} \DecValTok{2}\SpecialCharTok{:}\NormalTok{Nsim) \{}
\NormalTok{  Y }\OtherTok{=} \FunctionTok{runif}\NormalTok{(}\DecValTok{1}\NormalTok{)        }\CommentTok{\# proposed value from q}
\NormalTok{  alpha }\OtherTok{=} \FunctionTok{dbeta}\NormalTok{(Y, a, b) }\SpecialCharTok{/} \FunctionTok{dbeta}\NormalTok{(X[i}\DecValTok{{-}1}\NormalTok{], a, b)    }\CommentTok{\# q are all unif[0, 1] cancell out}
\NormalTok{  X[i] }\OtherTok{=}\NormalTok{ X[i}\DecValTok{{-}1}\NormalTok{] }\SpecialCharTok{+}\NormalTok{ (Y }\SpecialCharTok{{-}}\NormalTok{ X[i}\DecValTok{{-}1}\NormalTok{]) }\SpecialCharTok{*}\NormalTok{ (alpha }\SpecialCharTok{\textgreater{}} \FunctionTok{runif}\NormalTok{(}\DecValTok{1}\NormalTok{))   }\CommentTok{\# logical true or false}
\NormalTok{\}}

\FunctionTok{str}\NormalTok{(X)}
\end{Highlighting}
\end{Shaded}

\begin{verbatim}
##  num [1:5000] 0.686 0.236 0.236 0.379 0.379 ...
\end{verbatim}

\begin{Shaded}
\begin{Highlighting}[]
\FunctionTok{plot}\NormalTok{(X, }\AttributeTok{type =} \StringTok{"l"}\NormalTok{)    }\CommentTok{\# no pattern }
\end{Highlighting}
\end{Shaded}

\includegraphics{MH_sampler_files/figure-latex/unnamed-chunk-1-1.pdf}

\begin{Shaded}
\begin{Highlighting}[]
\FunctionTok{plot}\NormalTok{(X[}\DecValTok{4500}\SpecialCharTok{:}\DecValTok{4800}\NormalTok{], }\AttributeTok{type =} \StringTok{"l"}\NormalTok{)      }\CommentTok{\# for some intervals of time, the sequence (X(t)) does not change because all corresponding Y\textquotesingle{}s are rejected. }
\end{Highlighting}
\end{Shaded}

\includegraphics{MH_sampler_files/figure-latex/unnamed-chunk-1-2.pdf}
Remark:

\begin{itemize}
\tightlist
\item
  Those multiple occurrences of the same numerical value (rejected Y's)
  must be kept in the sample as such, otherwise, the validity of the
  approximation of f is lost!
\item
  Consider the entire chain as a sample, its histogram properly
  approximates the Be(2.7, 6.3) target.
\end{itemize}

\begin{Shaded}
\begin{Highlighting}[]
\FunctionTok{hist}\NormalTok{(X, }\AttributeTok{breaks =} \DecValTok{300}\NormalTok{)}
\FunctionTok{lines}\NormalTok{(}\FunctionTok{rbeta}\NormalTok{(}\DecValTok{5000}\NormalTok{, }\FloatTok{2.7}\NormalTok{, }\FloatTok{6.3}\NormalTok{), }\AttributeTok{col =} \StringTok{"red"}\NormalTok{)}
\end{Highlighting}
\end{Shaded}

\includegraphics{MH_sampler_files/figure-latex/unnamed-chunk-2-1.pdf}

Can checked even further using a Kolmogorov--Smirnov test of equality
between the two samples:

\begin{Shaded}
\begin{Highlighting}[]
\FunctionTok{ks.test}\NormalTok{(}\FunctionTok{jitter}\NormalTok{(X), }\FunctionTok{rbeta}\NormalTok{(}\DecValTok{5000}\NormalTok{, a, b))}
\end{Highlighting}
\end{Shaded}

\begin{verbatim}
## 
##  Two-sample Kolmogorov-Smirnov test
## 
## data:  jitter(X) and rbeta(5000, a, b)
## D = 0.0346, p-value = 0.005028
## alternative hypothesis: two-sided
\end{verbatim}

Can also compare the mean and variance

\begin{Shaded}
\begin{Highlighting}[]
\FunctionTok{mean}\NormalTok{(X)}
\end{Highlighting}
\end{Shaded}

\begin{verbatim}
## [1] 0.3027625
\end{verbatim}

\begin{Shaded}
\begin{Highlighting}[]
\FunctionTok{var}\NormalTok{(X)}
\end{Highlighting}
\end{Shaded}

\begin{verbatim}
## [1] 0.02085043
\end{verbatim}

\begin{Shaded}
\begin{Highlighting}[]
\CommentTok{\# theoretical }
\NormalTok{(mean\_theo }\OtherTok{\textless{}{-}}\NormalTok{ a}\SpecialCharTok{/}\NormalTok{(a}\SpecialCharTok{+}\NormalTok{b))}
\end{Highlighting}
\end{Shaded}

\begin{verbatim}
## [1] 0.3
\end{verbatim}

\begin{Shaded}
\begin{Highlighting}[]
\NormalTok{(var\_theo }\OtherTok{\textless{}{-}}\NormalTok{ a}\SpecialCharTok{*}\NormalTok{b }\SpecialCharTok{/}\NormalTok{ ((a}\SpecialCharTok{+}\NormalTok{b)}\SpecialCharTok{\^{}}\DecValTok{2} \SpecialCharTok{*}\NormalTok{ (a}\SpecialCharTok{+}\NormalTok{b}\SpecialCharTok{+}\DecValTok{1}\NormalTok{)))}
\end{Highlighting}
\end{Shaded}

\begin{verbatim}
## [1] 0.021
\end{verbatim}

Remark: - although rbeta output look similar as MH simulation output -
rbeta generates iid, but MH generates correlated samples. - so the
quality of the samples are degraded, and need more simulations to
achieve the same precision. - need the " effective sample size for
Markov chains" (Section 8.4.3).

\hypertarget{properties-of-mh}{%
\subsubsection{Properties of MH}\label{properties-of-mh}}

\begin{itemize}
\item
  In symmetric case, i.e.~\(q(y|x) = q(x|y)\), the acceptance probabilty
  alpha only depends on the ratio \(f(y)/ f(x^{(t)})\), so alpha is
  independent of q
\item
  But the performance of HM will be affected by the choice of q

  \begin{itemize}
  \tightlist
  \item
    if the supp(q) is too small, compared with the range of f, then the
    M chain will have difficulty to explore the range of f, and will
    coverge very slowly.
  \end{itemize}
\item
  Another property of MH algo: it only depends on the ratios:
  \(f(y)/ f(x^{(t)})\), \(q(x^{(t)})|y) / q(y|x^{(t)})\) hence
  independent of normalizing constant.
\item
  q may be chosen in a way that the intractable parts of f is canceled
  out.
\end{itemize}

\hypertarget{example-6.2}{%
\subsubsection{Example 6.2}\label{example-6.2}}

To generate a student-t random variable, (that is, when f corresponds to
a t(1) density), it is possible to use a N(0, 1) candidate within a
Metropolis--Hastings algorithm

\begin{Shaded}
\begin{Highlighting}[]
\NormalTok{Nsim }\OtherTok{\textless{}{-}}  \FloatTok{1e4}  
\NormalTok{X }\OtherTok{\textless{}{-}} \FunctionTok{rep}\NormalTok{(}\FunctionTok{runif}\NormalTok{(}\DecValTok{1}\NormalTok{), Nsim) }\CommentTok{\# intial value }
\ControlFlowTok{for}\NormalTok{ (i }\ControlFlowTok{in} \DecValTok{2}\SpecialCharTok{:}\NormalTok{Nsim) \{}
\NormalTok{  Y }\OtherTok{\textless{}{-}} \FunctionTok{rnorm}\NormalTok{(}\DecValTok{1}\NormalTok{)     }\CommentTok{\# proposal}
\NormalTok{  alpha }\OtherTok{\textless{}{-}} \FunctionTok{dt}\NormalTok{(Y, }\DecValTok{1}\NormalTok{) }\SpecialCharTok{*} \FunctionTok{dnorm}\NormalTok{(X[i}\DecValTok{{-}1}\NormalTok{]) }\SpecialCharTok{/}\NormalTok{ (}\FunctionTok{dt}\NormalTok{(X[i}\DecValTok{{-}1}\NormalTok{], }\DecValTok{1}\NormalTok{) }\SpecialCharTok{*} \FunctionTok{dnorm}\NormalTok{(Y))}
\NormalTok{  X[i] }\OtherTok{\textless{}{-}}\NormalTok{ X[i}\DecValTok{{-}1}\NormalTok{] }\SpecialCharTok{+}\NormalTok{ (Y }\SpecialCharTok{{-}}\NormalTok{ X[i}\DecValTok{{-}1}\NormalTok{]) }\SpecialCharTok{*}\NormalTok{ (alpha }\SpecialCharTok{\textgreater{}} \FunctionTok{runif}\NormalTok{(}\DecValTok{1}\NormalTok{))}
\NormalTok{\}}

\FunctionTok{str}\NormalTok{(X)   }\CommentTok{\# num [1:10000] 0.6923}
\end{Highlighting}
\end{Shaded}

\begin{verbatim}
##  num [1:10000] 0.00831 0.21775 -0.76952 1.07552 0.93699 ...
\end{verbatim}

\begin{Shaded}
\begin{Highlighting}[]
\FunctionTok{par}\NormalTok{(}\AttributeTok{mfrow =} \FunctionTok{c}\NormalTok{(}\DecValTok{2}\NormalTok{, }\DecValTok{2}\NormalTok{))}
\FunctionTok{hist}\NormalTok{(X, }\AttributeTok{breaks =} \DecValTok{250}\NormalTok{)}
\FunctionTok{acf}\NormalTok{(X)}

\CommentTok{\# want to  see the approximation to  P(X \textless{} 3)}
\FunctionTok{plot}\NormalTok{(}\FunctionTok{cumsum}\NormalTok{(X }\SpecialCharTok{\textless{}} \DecValTok{3}\NormalTok{) }\SpecialCharTok{/}\NormalTok{ (}\DecValTok{1}\SpecialCharTok{:}\FloatTok{1e4}\NormalTok{), }\AttributeTok{type =} \StringTok{"l"}\NormalTok{, }\AttributeTok{lwd =} \DecValTok{2}\NormalTok{)}
\end{Highlighting}
\end{Shaded}

\includegraphics{MH_sampler_files/figure-latex/unnamed-chunk-5-1.pdf}

\hypertarget{more-realistic-situation}{%
\subsection{More realistic situation}\label{more-realistic-situation}}

When the indept proposal q is derived from a preliminary estimation of
the parameters of the model. - the proposal could be a normal or t
distribution centered at the mle of theta and the variance - covariance
matrix be the inverse of fisher informaton matrix

\hypertarget{random-walk-mh}{%
\subsubsection{Random walk MH}\label{random-walk-mh}}

Example 6.4 formal problem of generating the normal distribution N(0, 1)
based on a random walk proposal equal to the uniform distribution on
{[}−δ, δ{]}.

\begin{Shaded}
\begin{Highlighting}[]
\NormalTok{Uni\_rdwk }\OtherTok{\textless{}{-}} \ControlFlowTok{function}\NormalTok{(delta) \{}
\NormalTok{  Nsim }\OtherTok{\textless{}{-}} \FloatTok{1e4}
\NormalTok{  X }\OtherTok{\textless{}{-}} \FunctionTok{rep}\NormalTok{(}\FunctionTok{runif}\NormalTok{(}\DecValTok{1}\NormalTok{), Nsim)}
  \ControlFlowTok{for}\NormalTok{ (i }\ControlFlowTok{in} \DecValTok{2}\SpecialCharTok{:}\NormalTok{Nsim) \{}
\NormalTok{    Y }\OtherTok{\textless{}{-}} \FunctionTok{runif}\NormalTok{(}\DecValTok{1}\NormalTok{, X[i}\DecValTok{{-}1}\NormalTok{] }\SpecialCharTok{{-}}\NormalTok{ delta, X[i}\DecValTok{{-}1}\NormalTok{] }\SpecialCharTok{+}\NormalTok{ delta)}
    \CommentTok{\# \textless{}{-} rnorm(1, X[i{-}1], 1)         \# proposal is N(X[i{-}1], 1) }
\NormalTok{    alpha }\OtherTok{\textless{}{-}} \FunctionTok{dnorm}\NormalTok{(Y) }\SpecialCharTok{/} \FunctionTok{dnorm}\NormalTok{(X[i}\DecValTok{{-}1}\NormalTok{])}
\NormalTok{    X[i] }\OtherTok{=}\NormalTok{ X[i}\DecValTok{{-}1}\NormalTok{] }\SpecialCharTok{+}\NormalTok{ (Y }\SpecialCharTok{{-}}\NormalTok{ X[i}\DecValTok{{-}1}\NormalTok{]) }\SpecialCharTok{*}\NormalTok{ (alpha }\SpecialCharTok{\textgreater{}} \FunctionTok{runif}\NormalTok{(}\DecValTok{1}\NormalTok{))}
\NormalTok{  \}}
\NormalTok{  X}
\NormalTok{\}}
\end{Highlighting}
\end{Shaded}

Calibrating the delta with 3 values: 0.1, 1, and 10

\begin{Shaded}
\begin{Highlighting}[]
\NormalTok{X\_0}\FloatTok{.1} \OtherTok{\textless{}{-}} \FunctionTok{Uni\_rdwk}\NormalTok{(}\FloatTok{0.1}\NormalTok{)}
\NormalTok{X\_1 }\OtherTok{\textless{}{-}} \FunctionTok{Uni\_rdwk}\NormalTok{(}\DecValTok{1}\NormalTok{)}
\NormalTok{X\_10 }\OtherTok{\textless{}{-}} \FunctionTok{Uni\_rdwk}\NormalTok{(}\DecValTok{10}\NormalTok{)}


\FunctionTok{par}\NormalTok{(}\AttributeTok{mfrow =} \FunctionTok{c}\NormalTok{(}\DecValTok{3}\NormalTok{, }\DecValTok{3}\NormalTok{))}

\CommentTok{\# plot cumsum}
\NormalTok{plt\_cum }\OtherTok{\textless{}{-}} \ControlFlowTok{function}\NormalTok{(X, ylim) \{}
  \FunctionTok{plot}\NormalTok{(}\FunctionTok{cumsum}\NormalTok{(X) }\SpecialCharTok{/}\NormalTok{ (}\DecValTok{1}\SpecialCharTok{:}\FloatTok{1e4}\NormalTok{), }\AttributeTok{ylim =}\NormalTok{ ylim)}
\NormalTok{\}}

\FunctionTok{plt\_cum}\NormalTok{(X\_0}\FloatTok{.1}\NormalTok{, }\AttributeTok{ylim =} \FunctionTok{c}\NormalTok{(}\SpecialCharTok{{-}}\DecValTok{1}\NormalTok{, }\DecValTok{1}\NormalTok{))}
\FunctionTok{plt\_cum}\NormalTok{(X\_1, }\AttributeTok{ylim =} \FunctionTok{c}\NormalTok{(}\SpecialCharTok{{-}}\DecValTok{1}\NormalTok{, }\DecValTok{1}\NormalTok{))}
\FunctionTok{plt\_cum}\NormalTok{(X\_10, }\AttributeTok{ylim =} \FunctionTok{c}\NormalTok{(}\SpecialCharTok{{-}}\DecValTok{1}\NormalTok{, }\DecValTok{1}\NormalTok{))}


\CommentTok{\# plot hist}
\NormalTok{plt\_hist }\OtherTok{\textless{}{-}} \ControlFlowTok{function}\NormalTok{(X) \{}
  \FunctionTok{hist}\NormalTok{(X, }\AttributeTok{breaks =} \DecValTok{250}\NormalTok{)}
\NormalTok{\}}

\FunctionTok{plt\_hist}\NormalTok{(X\_0}\FloatTok{.1}\NormalTok{)}
\FunctionTok{plt\_hist}\NormalTok{(X\_1)}
\FunctionTok{plt\_hist}\NormalTok{(X\_10)}



\CommentTok{\# Plot ACF}

\NormalTok{plt\_acf }\OtherTok{\textless{}{-}} \ControlFlowTok{function}\NormalTok{(X) \{}
  \FunctionTok{acf}\NormalTok{(X)}
\NormalTok{\}}

\FunctionTok{plt\_acf}\NormalTok{(X\_0}\FloatTok{.1}\NormalTok{)}
\FunctionTok{plt\_acf}\NormalTok{(X\_1)}
\FunctionTok{plt\_acf}\NormalTok{(X\_10)}
\end{Highlighting}
\end{Shaded}

\includegraphics{MH_sampler_files/figure-latex/unnamed-chunk-7-1.pdf}
Remark: - Too narrow or too wide a candidate (too small and too large
delta) results in slow convergence and high autocorrelation -
calibrating the scale δ of the random walk is crucial to achieving a
good approximation to the target distribution in a reasonable number of
iterations. - more realistic situations, this calibration becomes a
challenging issue

\hypertarget{adv-and-disadv-of-rd-walk}{%
\paragraph{Adv and disadv of Rd walk}\label{adv-and-disadv-of-rd-walk}}

\begin{itemize}
\tightlist
\item
  Idendpend MH only applies to some specific situations, while Rd walk
  caters to most cases
\item
  But Rd WALK is not the most efficient choic:

  \begin{itemize}
  \tightlist
  \item
    it requires many iterations for difficulities such the low
    probability regions between modal regions of f
  \item
    due to its simitray, it spends half the simulation revisit the
  \item
    So exists alternatives that bypass the perfect symmetry in the rdwk
    to gain efficiency
  \item
    although not always easy to implement.
  \end{itemize}
\end{itemize}

One of the alternatives is the Langivine that choose to move to heavier
value of target f by including the gradient in the proposal

But in the mixture model structure, the bimodal structure can be hard
for Lagenin as the local mode can be very attractive

\begin{Shaded}
\begin{Highlighting}[]
\NormalTok{like}\OtherTok{=}\ControlFlowTok{function}\NormalTok{(beda)\{}
\NormalTok{mia}\OtherTok{=}\FunctionTok{mean}\NormalTok{(Pima.tr}\SpecialCharTok{$}\NormalTok{bmi)}
\FunctionTok{prod}\NormalTok{(}\FunctionTok{pnorm}\NormalTok{(beda[}\DecValTok{1}\NormalTok{]}\SpecialCharTok{+}\NormalTok{(Pima.tr}\SpecialCharTok{$}\NormalTok{bm[Pima.tr}\SpecialCharTok{$}\NormalTok{t}\SpecialCharTok{==}\StringTok{"Yes"}\NormalTok{]}\SpecialCharTok{{-}}
\NormalTok{mia)}\SpecialCharTok{*}\NormalTok{beda[}\DecValTok{2}\NormalTok{]))}\SpecialCharTok{*}
\FunctionTok{prod}\NormalTok{(}\FunctionTok{pnorm}\NormalTok{(}\SpecialCharTok{{-}}\NormalTok{beda[}\DecValTok{1}\NormalTok{]}\SpecialCharTok{{-}}\NormalTok{(Pima.tr}\SpecialCharTok{$}\NormalTok{bm[Pima.tr}\SpecialCharTok{$}\NormalTok{t}\SpecialCharTok{==}\StringTok{"No"}\NormalTok{]}
\SpecialCharTok{{-}}\NormalTok{mia)}\SpecialCharTok{*}\NormalTok{beda[}\DecValTok{2}\NormalTok{]))}\SpecialCharTok{/}\FunctionTok{exp}\NormalTok{(}\FunctionTok{sum}\NormalTok{(beda}\SpecialCharTok{\^{}}\DecValTok{2}\NormalTok{)}\SpecialCharTok{/}\DecValTok{200}\NormalTok{)}
\NormalTok{\}}
\end{Highlighting}
\end{Shaded}


\end{document}
